\begin{frame}{What do households do?}
  
  \begin{itemize}
    \item Households maximize lifetime expected utility of consumption and labor
    \item We assume two types of households 
    \begin{itemize}
      \item \emph{Optimizing (OPT) households} that have access to financial markets and smooth their consumption across time
      \item \emph{Hand-to-mouth (HTM) households} that consume every period all its income
    \end{itemize}
    \item A share of HTM households is denoted $\omega$
  \end{itemize}
  
\end{frame}

\begin{frame}{Optimizing vs. HTM}
  
  \begin{itemize}
    \item Optimizing households
    \begin{itemize}
      \item Demand consumption and investment goods
      \item Supply labor to producers and government
      \item Accumulate and rent out capital to producers
      \item Issue debt, and pay interest
      \item Own producers and distributors
      \item Receive transfers, pay taxes
    \end{itemize}
    \item Hand-to-mouth households
    \begin{itemize}
      \item Demand consumption goods
      \item Supply labor to producers and government
      \item Receive transfers, pay taxes
    \end{itemize}
    
  \end{itemize}
\end{frame}

\begin{frame}{Aggregating OPT and HTM}
  \begin{itemize}
    \item Model is derived from micro-foundations
    \item We study behavior of each household separately
    \item $C^\opt_t$ and $C^\htm_t$ represents real consumption per household in respective cohort, not in absolute values
    \item Total consumption per household $C_t$ is an average
  \end{itemize}
  \vspace*{-2mm}
  \begin{equation*}
    C_t = \left(1-\omega\right)C_t^\opt+\omega\,C_t^\htm
  \end{equation*}      
  
\end{frame}

\begin{frame}{Technology}
  \begin{itemize}
    \item We want to construct a model where all real variables follow the same underlying productivity/technology growth (Balanced Growth Path) so real rations do not change in long-term (Steady-State)
    \item The growth of the trending overall technology $A_t$ is modeled by an AR(1) process 
    \vspace{-3mm}
    \begin{align*}
      \log\left(\gr A\right)= & \rho_{\text{A}}\,\log\left(\gr A_{t-1}\right) + 
      \left(1-\rho_{\text{A}}\right)\,\log\left(\sst{\gr A}\right)
    \end{align*}
  \end{itemize}
\end{frame}

\begin{frame}{Hand-to-mouth households}
  \begin{itemize}
    \item Since there's no saving and capital, the budget constraint is extremely simple
    \begin{align*}
      \Pc_t \, C^\htm_t & = \left(1 - \TRlit_t\right) W_t^\htm\, N^\htm_t \\
      & +  \TFgh^{\htm}_t  + \TFwh^{\htm}_t
    \end{align*}
    \item All real quantities in the budget constraint are again in per-capita units
    \item {We assume zero $TAXls^{\htm}$ as HTM cannot smooth it and private consumption would be unreasonably volatile}
  \end{itemize} 
\end{frame}

\begin{frame}{Optimizing households}
  
  \begin{itemize}
    \item Maximize lifetime expected utility subject to:
    \begin{itemize}
      \item Provide labor: receive wage
      \item Rent capital to producers: receive rent
      \item Issue private debt: pay interest
      \item Own companies: receive net profit
      \item Receive transfers (e.g. remittance inflow, government transfers, payments from energy sector, investment support)
      \item Pay taxes (Labor income tax, Lump-sum)
      \item Pay for consumption and investment goods
      \item Pay extra costs for changing investment level
    \end{itemize}
    
  \end{itemize}
  
\end{frame}

\begin{frame}{Maximize lifetime utility} 
  \vspace{-25mm}
  \begin{align*}
    & \qquad \qquad \qquad E_0 \sum_{t=0}^\infty \beta^t \, \mathcal{U}_t = \\
    & E_0\sum_{t=0}^\infty
    \expl{$\beta$}{subjective \\ time preference}{-95}{17mm}^t
    \bigg[ 
    \expl{$u_C \left( C^\opt_t \right)$}{utility \\ from consumption}{-75}{32mm} 
    + \expl{$u_N \left(N_t^{\opt} \right)$}{disutility \\ from work}{-80}{15mm}
    + \expl{{$u_V \left( V^{\opt}_t \right)$}}{utility \\ from net worth}{-85}{28mm}
    \bigg]
  \end{align*}
\end{frame}

\begin{frame}{Utility components}
  \vspace{-5mm}
  \small
  \begin{align*}
    u_C \left( C^\opt_t \right) & = \chi_0 {\log} \left( C^\opt_t {- \chi_{\DLI}\, \overline{\DLI}_t^{\, \opt}} - \chi_C \, \overline{C}^{\, \opt}_{t-1} dA_{ss} \right) \\
    u_N \left(N_t^{\opt}\right) & =  - \theta \frac{1}{1+\eta}\left(N_t^{\opt}\right)^{\left(1+\eta\right)} \\
    u_V \left(V^{\opt}_t\right) & =  {\nu  \left[\log\left(V^{\opt}_t\right) - \nu_0\, \frac{V^{\opt}_t}{\overline{C}^{\, \opt}_t }\right]}
  \end{align*} 
\end{frame}


\begin{frame}{Reference consumption level}
  \vspace{-5mm}
  \begin{itemize}
    \item Describes habit formation
    \item Penalizes consumption shifts 
    \item Households would like to keep at least a certain percentage of previous consumption typical across all households
    \begin{itemize}
      \item reference level $\overline{C}^{\, \opt}_{t-1}$
      \item reflecting e.g. a given social standard in the economy
      \item A particular household therefore does not optimize over the habit effect and it is considered exogenous (denoted with a "bar" above $C$)
      \item The percentage is given by $\chi_\cc$
    \end{itemize}
  \end{itemize} 
  
\end{frame}


\startframecont

\begin{frame}{Real disposable labor income}
  \begin{itemize}
    \item Represents current income reference level
    \item Makes consumption respond stronger to current income
    \item Why? In real world
    \begin{itemize}
      \item Households cannot smooth consumption effectively as local financial market is still shallow
      \item Even those who can smooth take current income as a reference level for their consumption (behavioral economics)
    \end{itemize}
    \item Key non-Ricardian and short-run crowding-in element
  \end{itemize} 
\end{frame}

\begin{frame}{Real disp. labor income}
  \begin{itemize}
    \item Deflated by consumption price index $P^\cc_t$
    \item Only selected income flows included in the definition - those that are perceived as a part of fundamental income
    \item Labor income after tax $\ W_t^{\opt} N_t^{\opt}\left(1- \TRlit_t\right)$
    \item Transfers from abroad $\TFgw^{\opt}_t$
    \item Transfers from government $\TFgh^{\opt}_t$
  \end{itemize} 
  \vspace{-5mm}
  {\small
  \begin{gather*}
    \DLI_t^\opt  = \frac{1}{P^\cc_t}\,\left[W_t^{\opt} N_t^{\opt}\left(1- \TRlit_t\right) +\TFwh^{\opt}_t + \TFgh^{\opt}_t\right]
  \end{gather*}
  }%
\end{frame}

\stopframecont

\startframecont  

\begin{frame}{Real net worth}
  \begin{itemize}
    \item Why? Introduced in literature as a short-cut for social status, inequality, precaution, myopia
    \item Current holdings of government bonds (through financial system) perceived as a part of wealth, irrespective of the future tax burden associated
    \item Key non-Ricardian and long-run crowding-out element
  \end{itemize} 
\end{frame}

\begin{frame}{Real net worth}
  \vspace{-10mm}
  \begin{itemize}
    \item Deflated by consumption price index $P^\cc_t$
    \item Physical capital own by households (denoted $\Kd_t$, $\Kz_t$ and $\Kh_t$ respectively)
    \item Net borrowings of households $\Bhb_t$ from the financial sector (=banking sector, financial intermediaries)
    \small
    \begin{gather*}
      P^\cc_t V_t^\opt  = \Pkd_t (\Kd_t^{\opt}+\Kh_t^{\opt})+ \Pkz_t \Kz_t^{\opt} + \Bhb_t^{\opt}
    \end{gather*}
  \end{itemize} 
\end{frame}

\stopframecont

\begin{frame}{Payment: Households' Debt}
  \small
  \begin{itemize}
    \item $\Rh_{t}$ is gross nominal interest rate payed by households for local borrowings
    \item $\Rh_{t} = 1.10$ means $10\%$ interest rate
    \item Households issue private debt to maximize lifetime consumption by borrowing from financial intermediaries
    \item Interest payed in period $t$ by households for local borrowings $\Bhb_{t}$ is 
  \end{itemize} 
  \vspace*{-2ex}
  \begin{gather*}
    Interest_t  = \left( \Rh_{t-1}-1 \right) \Bhb_{t-1} 
  \end{gather*}
  
\end{frame}


\startframecont

\begin{frame}{Capital accumulation}
  \begin{itemize}
    \item Households own two types of production capital: used for domestic production and for non-primary export production
    \begin{itemize} 
      \item Capital accumulation is the same for both types of capital $\Kd$ and $\Kz$ 
      \item We will use notation $\Kd$ in the following slides, but the equations for $\Kz$ would be the same
    \end{itemize}
  \end{itemize}
\end{frame}

\begin{frame}{Capital accumulation}
  \begin{itemize}
    \item $\Kd_t$ is the stock of capital \emph{at the end} of~period~$t$ that depreciates at a rate $\delta$
    \item Therefore the capital stock available for production in the current period is $\Kd_{t-1}$
    \begin{itemize}
      \item It represents a constraint for HH as today's investment becomes productive and yields a return only tomorrow (law of motion)
    \end{itemize}
  \end{itemize}
  \vspace*{-2ex}
  \begin{gather*}
    \Kd_{t} = \left(1 - \delta \right) \Kd_{t-1} + \Id_t
  \end{gather*}
\end{frame} 

\begin{frame}{Capital accumulation}
  \vspace{3mm}
  {\small
  \begin{gather*}
    \Xi^{\Id}_t =  \frac{1}{2}
    \expl{$\xi_{\Id}$}{Adjustment cost size}{100}{14mm}
    \,\overline{\Pi}_t\,\overline{\Id}_t\left(\log\left[ \left( \frac{\Id_t}{\sst{\gr A}\,\Id_{t-1}}\right)^{
    \expl{$\varphi_{\Id}$}{Weight of investment}{60}{11mm}
    }\,\left(\frac{\Id_t}{\Kd_{t-1}}\right)^{\left(1-\varphi_{\Id}\right)} \right] \right) ^2
  \end{gather*}
  }%
  \vspace{-7mm}
  \begin{itemize}
    \item Changing investment and capital brings capital-investment adjustment cost
    \item This cost is faced by individual households without having an aggregate effect
    \item The larger the adjustment cost size $\xi_{\Id}$, the more persistent investment will be
  \end{itemize}
\end{frame}

\stopframecont

\begin{frame}{Income: Capital}
  
  
  \begin{itemize}
    \item $\PRkd_{t}$ and $\PRkz_{t}$ are rental prices of capital 
    \begin{itemize}
      \item Rent represents an earning from renting out one unit of capital for one period (a year)
      \item Rent  motivates impatient households to postpone consumption and accumulate capital $\Kd$ and $\Kz$ in a view that future consumption will be higher
    \end{itemize}
    \item Change of capital stock requires time, so HH can rent only capital available earlier:
    \begin{gather*}
      Rent_t  = \PRkd_t\,\Kd_{t-1} + \PRkz_t\,\Kz_{t-1}
    \end{gather*}
  \end{itemize}
  
\end{frame}

\begin{frame}{Profit revenues}
  \begin{itemize}
    \item Optimizing households receive net profits from locally owned business
    \begin{itemize}
      \item Profit is generated by 
      \begin{itemize}
        \item local distributors and local producers $\PIEd_t$
        \item non-primary exporters $\PIEz_t$ 
        \item and by financial sector $\PIEb_t$
      \end{itemize}
      \item Profit is distributed on an aggregate level to all optimizing households and each individual household is too small to change the aggregate
    \end{itemize}
  \end{itemize}
\end{frame}


\begin{frame}{Budget constraint $\mathcal{B}_t$}
  \vspace{-5mm}
  \small
  \begin{gather*}    
    \Pc_t \, C^\opt_t + \Pii_t (\Id_t + \Iz_t +\Ih_t) \\
    + \Xi^{\Id}_t + \Xi^{\Iz}_t + \TAXls^{\opt}_t \\
    + \Bhb_t \\ = \\
    \Rh_{t-1}\,\Bhb_{t-1} \\
    + \left(1 - \TRlit_t\right) W_t\, N^\opt_t + \PRkd_t \, \Kd_{t-1} + \PRkz_t \, \Kz_{t-1} \\
    +  \TFgh^{\opt}_t + \TFwh^{\opt}_t + \TFgd^{\opt}_t + \TFgz^{\opt}_t \\
    + \PIEd_t + \PIEz_t + \PIEb_t + \Xi_t^{\text{total}}
  \end{gather*}
\end{frame}

\begin{frame}{FOCs}
  \begin{itemize}
    {\small
    \item The lifetime constrained optimization problem of the representative household 
    can be summarized as the following Lagrangian $\mathcal{L}_t$
    \item Maximum lifetime utility $\mathcal{U}_t$ with budget constraint $\mathcal{B}_t$ is reached when the first derivation of Lagrangian equals zero, so called \emph{first order conditions}
    }%
  \end{itemize}
  \vspace{-5mm}
  {\small
  \begin{align*}
    \mathcal{L}_t & = E\sum_{t=0}^\infty \beta^t \left[ \mathcal{U}_t - \Lambda^\opt_t \, \mathcal{B}_t \right] %\\ %\partial \mathcal{L}_t & = 0
  \end{align*}
  }%
\end{frame}

\begin{frame}{FOC: consumption}
  \vspace{-5mm}
  \small
  \begin{align*}
    \Lambda^\opt_t&\,P^\cc_t = \frac{\chi_0}{C^\opt_t - \chi_{\DLI}\,\DLI_t^\opt - \chi_\cc \,  C^\opt_{t-1}\,dA_{ss} } \\[5mm]
    % 1-\chi_{\DLI}\,\sst{\left[\frac{\DLI^\opt}{C^\opt}\right]} - \chi_\cc
    &\chi_0 = 1 \text{ to keep steady-state effects of $DLI$, or } \\
    &\chi_0 = 1-\chi_{\DLI}\,\sst{\left[\frac{\DLI^\opt}{C^\opt}\right]} - \chi_\cc
  \end{align*}
  \begin{itemize}
    \item Shadow price of income $\Lambda^\opt_t$ is equal to the marginal utility of consumption
    \item To get the additional utility of an extra nominal income in period $t$, multiply it~by~$\Lambda^\opt_t$
  \end{itemize}
\end{frame}



\begin{frame}{FOC: local borrowings}
  \begin{gather*}
    \expl{$\Lambda^\opt_t$}{Utility \\ loss today}{-140}{20mm} =
    \expl{$\beta$}{Today's value of the\\extra utility}{140}{25mm}\,
    \expl{$\Lambda^\opt_{t+1}$}{Effect on \\ tomorrow's utility}{-110}{15mm}\,
    \expl{$\Rh_t$}{Extra nominal \\ income tomorrow}{75}{20mm} + 
    \expl{$\nu\,\frac{\frac{C^\opt_t}{V_t}-\nu_0}{P^\cc_t\,{C^\opt_t}}$}{Wealth \\ effect}{-20}{25mm}
  \end{gather*}
  \vspace{10mm}
  \begin{itemize}
    \item The loss in utility of giving up some consumption and saving it is compensated by an extra consumption in the next period
  \end{itemize}
\end{frame}  


\begin{frame}{FOC: investment}
  \vspace{-5mm}
  \begin{itemize}
    \item Subject to investment in the local sector (same for non-primary export sector)
    \begin{align*}
      \small
      \Lambda_t\,\Pkd_t &= \Lambda_t\,\Pii_t\,\Big(1 + \Xi^{\Id}_t\Big) \notag\\
      & - \beta\,\Lambda_{t+1} \, \varphi_{\ii} \, \Pii_{t+1}\,\frac{\Id_{t+1}}{\Id_{t}} \, \Xi^{\Id}_{t+1}
    \end{align*}
    \item where the effect of the adjustment cost is:
    \vspace{-5mm}
    \begin{align*}
      \small
      \Xi^{\Id}_t &=  \xi_{\ii } \Big(\log\Id_t \notag\\
      &-(1-\varphi_{\ii})\log\left((\sst{\gr A} + \delta -1) \, \Kd_{t-1}\right) \notag\\
      &- \varphi_{\ii} \log \left( \sst{\gr A} \, \Id_{t-1}\right) \Big)
    \end{align*}
  \end{itemize}
\end{frame}   

\begin{frame}{FOC: capital}
  \begin{itemize}
    \item Subject to capital in the local sector (same for non-primary export sector)  
  \end{itemize}
  \small
  \begin{align*}
    \Lambda_t\,\Pkd_t &= \beta\,\Lambda_{t+1}\,\Big(\PRkd_{t+1} \notag \\
    &+ \Pkd_{t+1}\left(1 - \delta\right) + \left(1-\varphi_{\ii}\right) \Pii_{t+1}\,\frac{\Id_{t+1}}{\Kd_{t}} \, \Xi^{\Id}_{t+1}\Big)
  \end{align*}
\end{frame}

\startframecont
\begin{frame}{Financial intermediaries}
  \begin{itemize}
    \item only financial intermediaries (banking sector) can buy government LCY bonds on the primary market $\Bgb_t$ 
    \item households borrow from financial intermediaries $\Bhb_t$ (= borrowing net of deposits)
    \item financial intermediaries issue foreign private debt $\Bbw_t$ to close its position
    \item Budget constrain of financial intermediaries:
    \vspace{-3mm}
    \begin{align*}
      \Bhb_t + \Bgb_t = \Bbw_t
    \end{align*}
  \end{itemize}
\end{frame}

\begin{frame}{Financial intermediaries}
  \begin{itemize}
    \item Financial intermediaries generate profit $\PIEb$ which is distributed to optimizing households
    \item Revaluation effect of servicing external private debt is transferred to optimizing households
    \begin{align*}
      \PIEb_t = & (\Rh_{t-1}-1) \, \Bhb_{t-1} \\
      & + (\Rg_{t-1}-1) \, \Bgb_{t-1} \\
      & - (\Rh^{\star}_{t-1}-1) \, \frac{S_t}{S_{t-1}} \, \Bbw_{t-1}
    \end{align*}
  \end{itemize}
\end{frame}

\stopframecont

\begin{frame}{FOC: FCY loan}
  \begin{gather*}
    \expl{$-\Lambda^\opt_t$}{Extra utility \\ today}{-125}{16mm} =
    \expl{$\beta$}{Today's value of the\\ utility loss}{135}{23mm} \left( \,
    \expl{$-\Lambda^\opt_{t+1}$}{Effect on \\ tomorrow's utility}{-110}{15mm}\,
    \expl{$\Rh^{\star}_t \frac{S_{t+1}}{S_{t}}$}{Interest payment \\ tomorrow in LCY}{90}{20mm} \right)
    \expl{$-\nu\,\frac{\frac{C^\opt_t}{V_t}-\nu_0}{P^\cc_t\,{C^\opt_t}}$}{Wealth \\ effect}{-90}{17mm}
  \end{gather*}
  \vspace{10mm}
  \begin{itemize}
    \item The extra utility today is compensated by lower consumption in the next period
  \end{itemize}
\end{frame}  

\begin{frame}{FOC: UIP}
  \begin{itemize}
    \item The optimality condition for FCY loans simplifies to    
  \end{itemize}
  \vspace*{-3ex}
  \begin{align*}
    \Lambda^\opt_t =\beta\Lambda^\opt_{t+1} \Rh^{\star}_t \frac{S_{t+1}}{S_t} + \nu\,\frac{\frac{C^\opt_t}{V_t}-\nu_0}{P^\cc_t\,{C^\opt_t}}
  \end{align*}
  \vspace*{-2ex}
  \begin{itemize}
    \item implying   
  \end{itemize}
  \vspace*{-2ex}
  \begin{align*}
    \Rh_t = \Rh^{\star}_t \, \frac{S_{t+1}}{S_t} 
  \end{align*}
  \vspace*{-2ex}
  \begin{itemize}
    \item which is the Uncovered Interest Parity (UIP) condition
  \end{itemize}
\end{frame} 


% \begin{frame}{Payment: FCY Loan}
  %    \small
  %    \begin{itemize}
    %    \item $\Rh^{\star}_t$ is gross nominal interest rate on FCY borrowing of households
    %     \item Impatient households advance consumption by borrowing from abroad 
    %     \item $\Bhw_t$ represents a stock of FCY household debt expressed as its LCY equivalent
    %     \item Interest payment in period $t$ by households on their FCY borrowing (expressed in LCY equivalent)  
    %    \end{itemize} 
    %    \vspace*{-3ex}
    %     \begin{gather*}
      %      Interest_t = \left( \Rh^{\star}_{t-1}\frac{S_t}{S_{t-1}} -1 \right) \Bhw_{t-1} 
      %     \end{gather*}
      %  \end{frame}
      % - S_t \, \Rh^{\star}_{t-1} \, \frac{\Bhw_{t-1}}{S_{t-1}}\\
      % \item Save in government LCY bonds: receive interest
      % \item Issue private FCY bonds: pay interest
      % \item FCY borrowings $\Bhw_t$ from the rest of the world (expressed in LCY)
      % \item Government bonds $\Bgh_t$ issued locally
      
      